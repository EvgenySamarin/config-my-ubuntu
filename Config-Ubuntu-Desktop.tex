\documentclass[a4paper, 12pt]{report}

\usepackage{cmap} %Поиск в PDF
\usepackage[T2A]{fontenc} % кодировка
\usepackage[utf8]{inputenc} % кодировка исходящего текста
\usepackage[english,russian]{babel} % локализация и переносы
\usepackage{graphicx,wrapfig}
\graphicspath{ {./images/} }
\usepackage{placeins}
\usepackage{hyperref}
\usepackage{listings}
\usepackage{xcolor}

\definecolor{codegreen}{rgb}{0,0.6,0}
\definecolor{codegray}{rgb}{0.5,0.5,0.5}
\definecolor{codepurple}{rgb}{0.58,0,0.82}
\definecolor{backcolour}{rgb}{0.95,0.95,0.92}

\lstdefinestyle{mystyle}{
	backgroundcolor=\color{backcolour},   
	commentstyle=\color{codegreen},
	keywordstyle=\color{magenta},
	numberstyle=\tiny\color{codegray},
	stringstyle=\color{codepurple},
	basicstyle=\ttfamily\footnotesize,
	breakatwhitespace=false,         
	breaklines=true,                 
	captionpos=b,                    
	keepspaces=true,                 
	numbers=left,                    
	numbersep=5pt,                  
	showspaces=false,                
	showstringspaces=false,
	showtabs=false,                  
	tabsize=2
}

\lstset{style=mystyle}

\author{Евгений Самарин}
\title{Настройка desktop Ubuntu}
\date{\today\\v.1.0}

\begin{document}
\maketitle

\tableofcontents
\clearpage

\chapter{Введение}
\section{Предпосылки}

Я любитель новых сборок Ubuntu, как-то раз при очередном обновлении с версии 19.04 на 19.10 я столкнулся с проблемой переноса программ. Часть моих пакетов попросту была удалена из системы. И их пришлось устанавливать заново. А так как в процессе повторной установки можно что-то забыть и забытое как всегда напомнит о себе в самый неудобный момент, я буду вести список установленных мной программ. Возможно позже напишу скрипты для автоматической установки отдельных пакетов.

\chapter{Список программ}

\section{Системные программы и инструменты} 
\begin{center}
	\includegraphics[scale=0.5,angle=90]{gears}
\end{center} 
При переходе на новую сборку Ubuntu необходимо удостовериться, что все старые программы в наличии и работают должным образом. В этом могут помочь инструменты администрирования
\subsection{Менеджер пакетов Synaptic} Утилита для установки / удаления пакетов в графическом режиме, которые были установлены вручную из \lstinline|deb| файлов. Поможет подчистить "хвосты" приложений, которые невозможно удалить через стандартный менеджер. Не забудь, после удаления перезагрузить компьютер. Также поможет, если ты забыл название установленного пакета. В добавок покажет всё, что установлено на твоем ПК.
\subsection{apache2} Хотя он предустановлен в системе, как показывает практика настройки слетают при переустановке системы. Для начала проверь статус служб
\lstinline|service --status-all|, если служба \lstinline|apache2| не запущена - проверь настройки \lstinline|/etc/apache2/apache2.conf|. В прошлый раз у меня слетел \lstinline|phpmyadmin|, и из-за него не запускался \lstinline|apache2|.
\subsection{PHP} PHP необходим для развертывания локального web сервера, с целью тестирования приложений перед отправкой последних в PlayMarket. At that, I need different versions of a PHP. 5.6 for old projects and latest version for routine development respectively.
\lstinputlisting[language=bash]{./code/php.sh}
\subsection{MySQL server} Необходим для развертывания БД для web сервера.
\lstinputlisting[language=bash]{./code/mysql.sh}
 MySQL server версии 8 и выше использует auth2.0 в качестве способа аутентификации, однако один из моих проектов использует аутентификацию по паролю. Соответственно приходится менять настройки сервера.


\clearpage
\section{Необходимый минимум} 
\begin{center}
	\includegraphics[scale=0.6]{labor}
\end{center} 
Включает прикладные программные продукты, позволяющие выполнять основную деятельность на ПК.
\subsection{git} Удивительно, но git до сих пор не включен в базовый набор предустановленных программ для Ubuntu
\subsection{Postman} По возможности хотелось бы использовать эту программу, однако на моей сборке 19.10 она не запускается, что сильно печалит. Однако существует аналог, к примеру сейчас я обкатываю \lstinline|insomnia|
Нашел на SO способ починить постман. Правда осталась висящая ссылка в списке приложений, но тем не менее.
\lstinputlisting[language=sh]{./code/postman.sh}
После этих манипуляций программа будет запускаться через команду терминала \lstinline|postman|. Для удобства конечно следует создать ярлык для приложения.
Ярлыки хранятся в директории \lstinline|usr/share/applications|. Потребуется создать файл \lstinline|Postman.desktop| со следующим содержимым:
\lstinputlisting[language=sh]{./code/postman.desktop}
Не забудь сделать пере логин, иначе новый ярлык не подхватится.
\subsection{Vim} Мне нравится Vim как консольный редактор. Он легкий, плагинистый и настраиваемый. Текущий конфигурационный файл:
\lstinputlisting[language=ABAP]{./code/vimrc}
\subsection{fish} Автодополнение для терминала (работает с bash и zsh). 
Статья про использование fish: \href{https://habr.com/ru/post/248881/}{https://habr.com/ru/post/248881/}
\subsection{Ranger} удобный консольный файловый менеджер, с поддержкой всех shortcut из vim. Хвалебные оды про ranger:  \href{https://www.youtube.com/watch?v=0zkJG7iYTmc}{https://www.youtube.com/watch?v=0zkJG7iYTmc}
\subsection{DBeaver Community} СУБД для просмотра баз данных, использую его для просмотра sql таблиц.
\subsection{JetBrains toolbar} Нужен для обновления и установки моих рабочих инструментов: Android Studio, Intellij Idea. Для android studio требуется небольшая дополнительная настройка: \\
Установка плагинов: я выбрал для себя IdeaVim, ADB Wifi, MultiHighlights и DraculaTheme. В Vim используются сокращения, которые также используются в IDE, их необходимо разрешить:\\
\begin{list}{*}{}
	\item \lstinline|ctrl+c| - в vim используется не для копирования текста, а для сброса режимов до режима "command"/"normal". Копирование текста следует производить командой \lstinline|"+y| (в "визуальном" режиме для копирования в системный буфер) или командой \lstinline|y| (для копирования внутри текущего редактора). Вставку текста следует производить или командой \lstinline|p| или командой \lstinline|ctrl+r +| (для системного буфера - \textbf{работает в режиме вставки i}). Однако в vim существует также комбинация \lstinline|ctrl+[|, которая по сути делает тоже самое, поэтому следует использовать её. А в ide вне текста всё равно работает комбинация \lstinline|ctrl+c|, поэтому не имеет смысла использование комбинацию vim, оставим системный буфер в покое.\\ \textit{следует установить это значение в IDE}.
	\item \lstinline|ctrl+[| - как я писал выше, эта комбинация нам жизненно необходима, чтобы не тянуться бедным мизинчиком до клавиши \lstinline|esc|, ide в свою очередь использует это сочетание, чтобы прыгнуть в начало блока кода, тем более мы ничего не теряем, об этом ниже.\\ \textit{следует установить это значение в VIM}.
	\item \lstinline|ctrl+]| - раз уж мы переопределили предыдущее сочетание, не имеет смысла оставлять это за ide, как бонус в vim мы получим переход по ключевому слову под курсором. В данном контексте это будет дублированием комбинации \lstinline|ctrl+b|. В ide существует замена - сочетание клавиш \lstinline|ctrl+shift+m|\\ \textit{следует установить это значение в VIM}.
	\item \lstinline|ctrl+v| - в vim используется не вызова режима visual block, позволяет выделять часть текста блоком. Такого же эффекта можно добиться с помощью alt + мышь, однако вся соль состоит в том чтобы отказаться от мышки.\\ \textit{следует установить это значение в VIM}.
	\item \lstinline|ctrl+e| - в vim используется для прокрутки экрана вниз на одну строку, однако в ide есть более полезное назначение - открытие списка последних редактированных файлов.\\ \textit{следует установить это значение в IDE}.
	\item \lstinline|ctrl+u| - в vim используется для прокрутки на 1/2 экрана вверх. В ide используется, для перехода у super классу.\\ \textit{следует установить это значение в IDE}.
	\item \lstinline|ctrl+d| - в vim используется для прокрутки на 1/2 экрана вниз. В ide используется, для дублирования текущей строки.\\ \textit{следует установить это значение в VIM}.
	\item \lstinline|ctrl+y| - в vim используется для прокрутки экрана вверх на одну линию, в ide для удаления строки.\\ \textit{следует установить это значение в VIM}.
	\item \lstinline|ctrl+f| - в vim используется для постраничной прокрутки экрана вниз, в ide, для вызова поисковой строки по файлу.\\ \textit{Т.к. в vim существует собственный способ поиска, следует установить это значение в VIM}.
	\item \lstinline|ctrl+b| - в vim используется для постраничной прокрутки экрана вверх, в ide, для перехода к реализации базового класса.\\ \textit{следует установить это значение в IDE}.
	\item \lstinline|ctrl+g| - в vim используется чисто в информативных целях, для вывода названия текущего редактируемого файла, а также позиции на которой находится курсор. В ide, для перехода к указаной строке/столбцу. Я не вижу смысла ни в первом ни во втором (особенно для полноценной ОС).\\ \textit{следует установить это значение в VIM, только ради создания привычки}.
	\item \lstinline|ctrl+o| - в vim используется для перехода к предыдущей позиции курсора в стеке переходов, в ide, для вызова списка override методов.\\ \textit{В IDE сочетание поважнее, следует установить это значение в IDE}.
	\item \lstinline|ctrl+i| - в vim используется для перехода к следующей позиции курсора в стеке переходов, в ide, для вызова списка расширяемы методов.\\ \textit{В IDE сочетание поважнее, да и аналог есть (ctrl+shift+стрелки). Следует установить это значение в IDE}.
	\item \lstinline|ctrl+t| - в vim используется для перехода вниз по стеку меток, в ide, для работы с git - производит pull request. Стеком меток в vim я не пользуюсь, а вот git мне точно нужен.\\ \textit{следует установить это значение в IDE}.
	\item \lstinline|ctrl+k| - в vim используется в режиме вставки для ввода символов языков, отличных от латиницы, в которой vim работает по умолчанию. В ide используется, для работы с git - производит commit request.\\ \textit{следует установить это значение в IDE}.
	\item \lstinline|ctrl+l| - в vim используется для перерисовки экрана, в ide, для перехода к следующему найденому фрагменту. Я не буду пользоваться ни первым ни вторым.\\ \textit{следует установить это значение в VIM}.
	\item \lstinline|ctrl+r| - в vim одно из самых полезных сочетаний - работает как Redo, для восстановления отменённых изменений (необходимо для восстановления данных из системного буфера обмена в режиме "вставки"). В ide используется, для замены текста.\\ \textit{следует установить это значение в VIM}.
	\item \lstinline|ctrl+w| - в vim используется для перемещения между split окнами, в IDE для расширения выделенной области. \textbf{Обе фишки полезные не знаю, что выбрать}.\\ \textit{ещё подумаю, пока следует установить это значение в VIM}.
	\item \lstinline|ctrl+x| - не разобрался для чего это нужно, приведу статью из учебника.
	\begin{quotation}\textit{"Enter CTRL-X mode.  This is a sub-mode where commands can	be given to complete words or scroll the window. See |i\_CTRL-X| and |ins-completion|. \{not in Vi\}"}\\
	Информация с сайта:\\ \href{http://vimdoc.sourceforge.net/htmldoc/insert.html}{http://vimdoc.sourceforge.net/htmldoc/insert.html}
	\end{quotation}
	Есть негласное правило, если не знаешь как работает фича - не используй её. \textit{следует установить это значение в IDE}.
	\item \lstinline|ctrl+a, ctrl+shift+2| - не разобрался, что делают эти hotkey.\\ \textit{следует установить это значение в IDE}.
	\item \lstinline|ctrl+p| - в vim используется в режиме вставки как autocomplete, в ide используется, для отображения параметров методов. По понятным причинам ide лучше.\\ \textit{следует установить это значение в IDE}.
	\item \lstinline|ctrl+s| - в vim используется для "заморозки" терминала при этом работа приложения не завершается, все изменения продолжают происходить (выйти из режима можно комбинацией \lstinline|ctrl+q|), в ide используется для сохранения изменений. Эта фича работает только в консоли.\\ \textit{следует установить это значение в IDE}.
	\item \lstinline|ctrl+n| - в vim используется для авто подстановки слов, в ide используется в различных ситуациях, по умолчанию открывает строку поиска, хотя также используется для создания нового файла, впрочем как и комбинация \lstinline|alt+insert|. По этому проще убрать комбинацию с создания файла в keymap.\\ \textit{следует установить это значение в IDE, всё равно не получиться вызвать vim map для этой комбинации, будет работать только в консоли}.
	\item \lstinline|ctrl+m| - в vim используется для перехода к первому не пробельному символу в следующей строке, в ide используется для быстрого создания функции из выражения, я этой штукой не пользуюсь.\\ \textit{следует установить это значение в VIM}.
	\item \lstinline|ctrl+h| - в vim используется как альтернатива backspase, в ide для просмотра иерархии класса.\\ \textit{следует установить это значение в IDE}.
	\item \lstinline|ctrl+2| - в vim используется в режиме вставки, для вставки последней введенной последовательности символов. Не нужная на мой взгляд фича.\\ \textit{следует установить это значение в IDE}.
	\item \lstinline|ctrl+shift+6| - в vim используется для перехода в предыдущий файл, на позицию курсора. Есть альтернатива в IDE.\\ \textit{следует установить это значение в IDE}.
\end{list}
Для MultiHighlights нужно добавить hotkeys. Для закрепления выделения за словом использую сочетания клавиш CTRL+Quote, для снятия выделения со всех слов - ALT+CTRL+Quote.\\
Также следует добавить несколько \lstinline|LiveTemplates|. Область видимости включаем Kotlin. Переменная \lstinline|date| во всех случаях настроена след образом: \lstinline|name=date, expression=date("yyyy.MM.dd"), default=, skip if defined=true|.\\
Все элементы создаем в списке \lstinline|AndriodComments|\\\\
Шаблоны для быстрого создания описаний:\\
Аббревиатура шаблона: \lstinline|descm|
Описание: \lstinline|add method description|
\lstinputlisting[language=java]{./code/lt-descm.kt}
Аббревиатура шаблона: \lstinline|desc|
Описание: \lstinline|add JavaDoc description|
\lstinputlisting[language=java]{./code/lt-desc.kt}
Шаблоны для быстрого создания \lstinline|todo| напоминаний.\\
Аббревиатура шаблона: \lstinline|todo|
Описание: \lstinline|adds // TODO|
\lstinputlisting[language=java]{./code/lt-todo.kt}
Далее добавляем шаблоны по аналогии с todo для остальных напоминалок.

ADB Wifi не требует настройки, зато требует определённых манипуляций для работы.
\begin{enumerate}
	\item Подключаем девайс по USB
	\item Открываем терминал в папке с adb по пути \lstinline|SDK-PATH/sdk/platform-tools|
	\item Набираем \lstinline|adb tcpip 5555|
	\item Набираем \lstinline|adb connect DEVICE-IP|
	\item Отключаем USB
\end{enumerate}
Теперь можно собирать программы на устройство без необходимости подключения по кабелю.

\subsection{Менеджер виртуальных машин} Очень редко нужен, последний раз использовал, чтобы подобрать для себя VDS сервер. Однако, как показывает практика пакет нужен для работы эмулятора Android.
\subsection{Enpass} Для хранения паролей и другой конфиденциальной информации.


\clearpage
\section{Дополнительные и полезные штуки} 
\begin{center}
	\includegraphics[scale=0.3]{comics}
\end{center} 
Программы и пакеты которые не критичны для работы, но их присутствие скрашивает серые будни линусоида.
\subsection{Latex}
\textit{Thanks, YouTube channel Диджитализируй for show me this perfect PDF creating product.}
В качестве редактора используем \textbf{TexStudio}. Кстати, эту статью я пишу как раз с использованием этой студии, эдакая работа с текстом для программистов. Тебе не нужно думать над подбором стилей текста и тому подобного. Ты просто пишешь текст а он выглядит потрясающе. Подобный документ выходит после компиляции исходного текста, да да - нужно компилировать набранный текст.
\lstinputlisting[language=bash]{./code/latex.sh}
\subsection{twinux} Twitter client - для написания постов и чтения новостей.
\subsection{calibre} Программа для каталогизации книг. Эдакая личная книжная полка
\subsection{Rclone Browser} Для синхронизации электронной книжки "по воздуху" через Dropbox, а также для связи с другими облаками: Google, Yandex, OneDrive
\subsection{Steam} Для игрушек, что поделать - все мы не безгрешны.
\subsection{Discord} Для тех же игрушек, чтобы общаться с друзьями.
\subsection{PhotoGimp} Аналог Photoshop для Linux.

\chapter{Поднастройка окружения}
\section{Замена местами кнопок CAPS / CTRL} Я пользователь VIM и мне часто приходится нажимать \lstinline|ctrl+[|, стоит пожалеть свой мизинец и не тянуться в судорожных попытках к клавише ctrl - это облегчит жизнь и написание кода. За основу взял статью Хабра: \href{https://habr.com/ru/post/222285/}{https://habr.com/ru/post/222285/} о перенастройке клавиш в unix системах. Первое что нам потребуется - утилита \lstinline|xkbcomp|.
Выполняем чтение текущих настроек системы \lstinline|setxkbmap -layout us,ru -print|. Записываем то что вывелось в файл my и размещаем его в директорию \lstinline|~/.config/xkb/my|. Теперь открываем файл на редактирование, в нём будем прописывать конфигурации биндинга клавиш.
Утилита \lstinline|xkbcomp| из коробки уже имеет ряд стандартных конфигов, которые можно включить прописав \lstinline|include "capslock(escape)" | в блок \lstinline|xkb_symbols "my" { your_code_here }|, где capslock - название конфигурационного файла из списка \lstinline|/usr/share/X11/xkb/symbols|, а escape - название конфигурационного параметра. В директории symbols все параметры любезно описаны разработчиками и готовы к употреблению. 
Итак, мой файл конфигурации выглядит след образом: 
\lstinputlisting[language=sh]{./code/my-xkb}
Запустить конфигурацию можно командой:\\ \lstinline|xkbcomp $HOME/.config/xkb/my $DISPLAY|\\

\begin{quote}
\textbf{Важно:} если что-то пойдет не так, как задумывалось, сбросить настройки клавиш можно вызвав команду \lstinline|setxkbmap| без параметров.
\end{quote}.

В общем на новых системах Gnome этот способ работает криво. будем довольствоваться заменой левого контрола на caps и наоборот. Необходимо скачать gnome-tweak-tools (дополнительные настройки gnome) и в раздале клавиатура выбрать "поменять местами ctrl и capslock". И конфиги писать не нужно.

\section{Перевод выделенного текста по shortcut внутри системы} Будет работать при наличии интернета, так как используется запрос к google translate API. Итак, устанавливаем пакеты \lstinline|libnotify-bin| (благодаря ему скрипт сможет отображать уведомления на рабочем столе), \lstinline|wget| (для получения перевода от Google) и \lstinline|xsel| (нужен для извлечения выделенного текста)\\

\lstinline[language=sh]{sudo apt-get install libnotify-bin wget xsel}\\

Создаем файл с любым названием в домашнем каталоге, например "notitrans", внутри файла пишем скрипт:

\lstinputlisting[language=sh]{./code/notitrans}

В тексте скрипта, приведенном выше, меняем «tl=en» на обозначение языка, в котором хотелось бы получать результат перевода

Теперь делаем файл исполняемым:\\
\lstinline[language=sh]{chmod +x ~/notitrans}\\

Перемещаем скрипт в каталог \lstinline|/usr/local/bin/|:\\
\lstinline[language=sh]{sudo mv ~/notitrans /usr/local/bin/}\\

Теперь биндим удобные клавиши для перевода:
В GNOME (и Unity) это можно сделать перейдя в \lstinline|System Settings > Keyboard > Shortcuts > Custom Shortcuts| и нажав «+» для добавления нового сочетания клавиш. В качестве shortcut я выбрал \lstinline|ctrl+shift+>|. Готово, наслаждаемся встроенным в систему переводчиком.

\section{Замена некоторых стандартных shortcut на более удобные}
Замена shortcut не многим отличается от способа описанного в предыдущем параграфе:

В GNOME (и Unity) это можно сделать перейдя в \lstinline|System Settings > Keyboard > Shortcuts| в поиске набираем нужную нам команду.

\subsection{Снимок выделенной области на экране в буфер обмена}
Стандартное сочетание \lstinline|ctrl+shift+PrintScreen| не удобно пальцы страдают, меняем на \lstinline|ctrl+shift+S| 

\end{document}
